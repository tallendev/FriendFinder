\documentclass[11pt]{article}
\bibliographystyle{plain}
\usepackage{graphicx}
\usepackage[margin=.75in]{geometry}
\usepackage{amsmath}
\usepackage{comment}
\usepackage{setspace}
\usepackage{indentfirst}
\usepackage{hyperref}
\usepackage{url}
\doublespace

\begin{document}
\title{CS496 Proposal: FriendFinder}
\author{Tyler Allen}
\maketitle

% What is it
\section{Overview}

The goal of this capstone is to develop a social networking application for 
the Android mobile platform. The application will allow users to connect with other people and 
plan events. Users will be able to find people by name or through mutual interests. The application
will allow users to invite other users to events they are planning. The application will display 
friends and nearby users on a map. The application will allow users to select 
a location for their events and have it appear on a map to other users.
This will provide students with a fun and easy-to-use social networking
application.\\

\section{Semester One Background}
The application currently supports basic a basic set of functionality including
user registration and authentication, secure network communication between 
a backend server and instances of the client application, and the ability for users 
to search and view groups and events. Users have their own profiles that contains some basic 
information about them and their interests. This includes their name, birthdate, 
gender, and e-mail address. Users can also navigate through the full application,
despite some features not being implemented yet. All data is stored on a backend database in order
to synchronize data and prevent unauthorized access of some user data. The backend
database is accessed by the client application using Transfer Control Protocol (TCP)
for network communication and Secure Socket Layer (SSL) for the encryption of 
network data. JavaScript Object Notation (JSON) is used as the protocol for the 
transfer of application data. Java Database Connectivity (JDBC) is used to 
access the backend PostgreSQL database. A user-relevant subset of the backend data is 
stored within the application using a Sqlite3 database. \\

\section{Semester Two}
The application will allow users to create events and groups, as 
well as invite other users to events. Users will also be able to modify their group, event, and 
profile information. In addition, users should be able to upload profile pictures to represent 
themselves.\\

Users will be able to upload a schedule using the Android Calendar framework\cite{calendar}.
Users will be able to keep a list of friends; a user may view their friend's schedule through their 
user profile. To preserve privacy, other users will be able to view if a user is currently busy, 
according to their schedule, by looking at their profile.\\

Event pages will allow users to view the location of a scheduled event using a Google Map\cite{googlemaps}.
Users will also be able to view the location of their friends, relative to their own, on a 
Google Map\cite{googlemaps} interface. The ability to view locations should be relative to a user's 
position, and extend to a certain radius around that area. Events searches may be limited to a 
location radius as well. All Google Maps\cite{googlemaps} and location-based information will 
require a way to transmit location data over the network and store it in a database, or a 
peer-to-peer system for transmitting this data to other users\cite{location}.\\

\section{Testing}
With the implementation of a variety of new functionality and frameworks, it is critical that the
application is regularly tested for regression. Both the application and server backend will require 
some level of unit testing to be implemented to prevent regression. This will include the user of 
the standard Java unit testing suite, JUnit\cite{junit}. The application will make use of the 
standard Android testing framework, which is based on JUnit\cite{androidtest}. Android 
test devices include an LG G2, an HTC One M8, a Nexus 7, and a Nexus 10 tablet. 
Testing will include the aesthetic quality of layouts, the functionality of layouts,
the existance of error checking in the event of unpreventable issues such as 
network connection loss, and the proper and safe querying, modification, and insertion
of database entries. Testing will performed during and upon completion of each 
feature, with Unit Tests being developed incrementally as time allows.


%Why is it a good capstone
\section{Problem Statement}

The remaining functionality to be implemented contain a number of challenging aspects. Presently, 
the backbone of the project is complete; database and network functionality are implemented. However,
allowing users to modify profiles, events, and groups requires the implementation of several database
functionalities. Allowing clients the ability to modify backend database data requires additional
precautions and careful exposition of functionality in order to prevent inadvertent data destruction. 
Research will need to be completed on the functionality of the Android testing and calendar frameworks.
Allowing users to share current location information with their friends will require the ability
to store location data on the server. Alternatively, it may prove to be too slow, or data intensive,
so a peer to peer option needs to be investigated for this setting. Applications with a feature similar to this, such as YikYak\cite{yikyak},
will need to be investigated for this purpose. Unit Testing on mobile devices may also prove difficult when
it comes to testing user interface elements.


\newpage
\section{Requirements Specification - MoSCoW Analysis}

\subsection{Must Have}
\begin{itemize}
\item User creation of groups and events.
\item User modification of profile, groups, and events.
\item Map locations of events with markers and touch-event informational pop-up windows.
\item User schedules and a 'busy/available' status on user profiles..
\item Providing an interface for visualizing locations of friends.
\item Extension of user location visualization to events.
\end{itemize}
\subsection{Should Have}
\begin{itemize}
\item User profile pictures.
\item Ability for users to enter their own likes instead of selecting from a dropdown menu.
\item Provide a system for suggesting a like that another user has already created. 
This would likely be implemented as an autocomplete function.
\end{itemize}
\subsection{Could Have}
\begin{itemize}
\item An ``I forgot my password'' button.
\item Token authentication.
\item Autocomplete for searching and other text fields.
\end{itemize}
\subsection{Won't Have}
\begin{itemize}
\item Group recommendations.
\item Web application.
\item IPhone implementation.
\end{itemize}
\newpage
\section{Schedule of Completion}

\textbf{Tuesday, February 3, 2015}
\\
Implementation and testing of group, event creation and modification will be completed.
Users will be able to create and modify groups and events.

\textbf{Tuesday, February 17, 2015}
\\  
Events will support locations and display them on a map. This includes a 
method for storing and parsing location data. Users will also be able to select
times and dates from dialogs on their events. User profile modification 
will be completed. Testings of maps and user profile modification will be 
complete.

\textbf{Tuesday, March 3, 2015}
\\	
Users will be able to maintain a user's schedule inside the application on
a calendar interface. Busy/Available statuses will be visible on user profiles.
Testing of user schedules and statuses will be completed.

\textbf{Tuesday, March 24, 2015}
\\
Users will be able to visualize locations of their friends. Peer to peer transfer will be
investigated, and possibly implemented. Unit tests will be caught up to this point.
 
\textbf{Tuesday, April 7th, 2015}
\\
User and group pictures will be implemented. Users will be able to submit 
photos for their group or user account for others to see. Storage of profile pictures on 
server and client side will be implemented. Testing of picture storage and 
retrieval will be completed.
 
\textbf{Tuesday, April 21, 2014}
\\	
Autocomplete for search fields will be complete. An ``I forgot my password'' button will 
be available and functional. Users will be able to reset their password from 
within the application. Testing of autocomplete and ``I forgot my password'' 
functionality will be completed.


\newpage
\bibliography{capstone_proposal}
\nocite{}
\end{document}
